\documentclass[a4paper,11pt]{article}

% define the title
\author{Haoyu Lin}
\title{Lennard-Jones Potential}
\begin{document}

% generates the title
\maketitle

% insert the table of contents
%\tableofcontents

\section{Lennard-Jones Potential}
A Lennard-Jones (LJ) potential $V(r_{ij})$ between two atoms placed at distance $r_{ij} = |\mathbf{r}_j - \mathbf{r}_i|$ is defined by:
\begin{equation}
    V(r_{ij}) = 4 \epsilon \left[ \left( \frac{\sigma}{r_{ij}} \right)^{12} - \left( \frac{\sigma}{r_{ij}} \right)^{6} \right].
\end{equation}
And the force on atom $i$ produced by LJ potetial is
\begin{equation}
    \mathbf{F}_i = - \frac{24 \epsilon}{r_{ij}} \left[ 2\left( \frac{\sigma}{r_{ij}} \right)^{12} - \left( \frac{\sigma}{r_{ij}} \right)^{6} \right] \hat{\mathbf{r}}_{ij},
\end{equation}
where
\begin{equation}
    \hat{\mathbf{r}}_{ij} = \frac{\mathbf{r}_{ij}}{r_{ij}} = \frac{x_{ij}\hat{\mathbf{x}} + y_{ij}\hat{\mathbf{y}} + z_{ij}\hat{\mathbf{z}}}{\sqrt{x_{ij}^2 + y_{ij}^2 + z_{ij}^2}}.
\end{equation}
Similarly, the force on atom $j$ is
\begin{equation}
    \mathbf{F}_j = \frac{24 \epsilon}{r_{ij}} \left[ 2\left( \frac{\sigma}{r_{ij}} \right)^{12} - \left( \frac{\sigma}{r_{ij}} \right)^{6} \right] \hat{\mathbf{r}}_{ij},
\end{equation}
which is just opposite to $\mathbf{F}_i$.

\end{document}